\section{Exponential Smoothing}\label{sec:exponential-smoothing}

As previously discussed, the dataset does not incorporate any seasonality and the dataset exhibits a near-linear
trend with a deviation towards the end.
This leads us to the (A,N) model in ETS terminology, where the trend is additive and the seasonality
is none, which is also known as Holt's linear trend method.\\

Upon fitting the model, we proceeded to analyze the residuals.
Residuals are the differences between our observed and predicted values, and their analysis is crucial for
understanding any patterns or systematic structures that the model may have failed to capture.
To evaluate the accuracy of the model, we calculated the previously established accuracy metrics -
Mean Absolute Error (MAE) and Root Mean Squared Error (RMSE).
The model yielded a MAE of 196.67898 and a RMSE of 233.5003 on the test set.
The respective values on the training set are much lower, which might indicate that the model is overfitting the training set.\\

To ensure the adequacy of our model, we conducted a Ljung-Box test on the residuals.
The test yielded a Q* statistic of 37.732 with 24 degrees of freedom, resulting in a p-value of 0.03692.
Since the p-value is less than the conventional threshold of 0.05, it suggests the presence of autocorrelation in the residuals at lag 24.
This indicates that there might be some information left in the residuals that the model has not captured.
As we picked the model based on the ETS framework, we assume that the model is adequate.\\

The performance of the model compared to the simple model described in the previous section is lower.
The simple model yielded a MAE of 123.95944 and a RMSE of 152.2772 on the test set.
We assume that our selected exponential smoothing model has an adequate ACI value, as we picked the model
based on the ETS framework, which is why we did not consider any other models.
