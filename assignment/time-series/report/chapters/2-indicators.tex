\section{Performance Indicators}\label{sec:performance-indicators}

In order to evaluate the efficiency of forecasting models applied to the dataset concerning
"Commercial paper outstanding, financial companies", it is pivotal to establish relevant performance indicators.
These indicators provide quantifiable metrics to assess the accuracy and reliability of the forecasts generated by the models.
After a thorough analysis, two prominent indicators have been selected: \textbf{Mean Absolute Error (MAE)} and \textbf{Root Mean Square Error (RMSE)}.

\subsection{Mean Absolute Error (MAE)}\label{subsec:mae}

The Mean Absolute Error (MAE) is a well-regarded indicator that measures the average magnitude of errors between the
forecasted values and the actual values, without considering the direction of the errors.
This metric is straightforward and provides a clear depiction of the forecasting model's accuracy in numeric terms.
It is particularly useful for our dataset as it offers a clear interpretation of the average error magnitude,
which is crucial for assessing the model's performance in forecasting commercial paper outstanding.\\

A lower value of MAE is preferable as it signifies a lower average error in forecasts.
However, since MAE is scale-dependent, we need to incorporate the scale of the data to interpret the value of MAE.
For this analysis, a target of $\text{MAE} < 80$ has been set, aiming for a minimal deviation between the forecasted and actual values.

\subsection{Root Mean Square Error (RMSE)}\label{subsec:rmse}

The Root Mean Square Error (RMSE) is another robust indicator that also measures the average magnitude of errors between the forecasted and actual values.
Unlike MAE, RMSE squares the errors before averaging, which places a higher weight on large errors.
This characteristic makes RMSE a valuable indicator for our dataset, especially when large deviations from actual values are undesirable.\\

Similar to MAE, a lower value of RMSE is desired as it indicates a lower level of error in the forecasts.
RSME is also scale-dependent therefore, the scale of the data needs to be considered when interpreting the value of RMSE same as with MAE.
A target of $\text{RMSE} < 180$ has been set for this analysis, striving for a balance between penalizing large errors and achieving an acceptable level of forecasting accuracy.
