\section{Selection and Evaluation of Forecasting Model}\label{sec:selection-and-evaluation-of-forecasting-model}

Given the nature of the dataset concerning "Commercial paper outstanding, financial companies", selecting an appropriate simple forecasting model is crucial for accurate predictions.
The dataset exhibits a near-linear trend with a deviation towards the end, characteristic of many financial time-series data.
Among the simple models discussed - Average Method, Naive Method, Seasonal Naive Method, and Drift Method - the Drift Method, also known as Random Walk with Drift, and the Random Walk without drift are considered relevant for this analysis.\\

The Naive Method, although simplistic, may not capture the trend inherent in the data.
Similarly, the Average Method is likely to overlook the evolving trend as it would compute a mean that falls somewhere between the earlier lower values and the later higher values, thereby providing inaccurate forecasts.
The Seasonal Naive Method is not suitable due to the absence of a seasonal pattern in the data.\\

The Drift Method, which is essentially a Random Walk with Drift, incorporates a constant term to account for a linear trend in the data.
However, given the deviation from the linear trend towards the end of the series, a Random Walk without drift appears to be a more fitting choice.
This model is simplistic yet relevant for financial time-series data exhibiting a near-linear trend with some deviation.

\subsection{Evaluation of Forecasting Model}\label{subsec:evaluation-of-forecasting-model}

After fitting the Random Walk without drift model to the dataset, a comprehensive analysis of the residuals and the model's performance on the test set was conducted
to evaluate its forecasting accuracy and reliability.\\

The residuals from the model were analyzed using the Ljung-Box test to check for auto-correlation.
The test yielded a Q* statistic of 38.88 with 24 degrees of freedom, resulting in a p-value of 0.02811.
This p-value, being below the conventional threshold of 0.05, indicates the potential presence of auto-correlation in the residuals.
It may be beneficial to further explore the residual behavior through additional diagnostic tests or visual inspections to ensure the model's assumptions are adequately met.\\

The performance of the model on the test set was assessed using the previously established indicators - Mean Absolute Error (MAE) and Root Mean Square Error (RMSE).
The model yielded a MAE of 123.95944 and a RMSE of 152.2772.
These values are within the target thresholds of 80 and 180, respectively, indicating that the model is reasonably accurate in forecasting the commercial paper outstanding.\\

To further evaluate the model's performance, we performed several accuracy checks on other models, such as the naive method, the average method, seasonal naive method, and the drift method.
It was observed that the values of MAE and RMSE for the seasonal naive method were lower than the random walk without drift model.
However, the seasonal naive method is not suitable for this dataset as it does not account for the linear trend in the data.
We can conclude that the random walk without drift model is the most appropriate model for this dataset, given its simplicity and accuracy in forecasting the commercial paper outstanding.
