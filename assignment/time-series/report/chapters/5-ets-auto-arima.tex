\section{ETS and Auto ARIMA}\label{sec:ets-and-auto-arima}

Now we apply the ETS and Auto ARIMA models to the dataset.
This clarifies the best model for the dataset and allows us to compare the results to the previously established models.\\

The ETS model suggests a ETS(M,A,N) model, which is equivalent to Holt's linear trend method.
The model yielded a MAE of 198.0852 and a RMSE of 235.04577 on the test set.
These values are slightly higher than the values of the simple model described in the previous section.
To further elaborate on the results and to ensure that the ETS model is adequate, we compared the AIC values of the models.
We found that the SES model has a AIC value of 1593.299 and the ETS model has a AIC value of 1564.055.
This indicates that the ETS model is more adequate than the SES model, even though the performance of the ETS model is lower.
This is due to the fact that these error metrics are not always a good indicator of the adequacy of a model.\\

The Auto ARIMA Model suggests a ARIMA(0,1,0)(1,0,0)[12] with drift model.
The model yielded a MAE of 265.81343 and a RMSE of 314.24356 on the test set.
These values are much higher than the values of the simple model described in the previous section.
We also compared the AIC values of the models.
We found that the Auto ARIMA model has a AIC value of 1311.64, which is much lower than the AIC values of the other models.
This indicates that the Auto ARIMA model is more adequate than the other models.\\

Conducting further analysis using the Ljung-Box test on the residuals of the models, we found that the the residuals of the ETS model showed a Q* statistic of 16.13 with 24 lags and a p-value of 0.8496, suggesting that the residuals are close to white noise. Similar results were found for the Auto ARIMA model, with a Q* statistic of 32.015 with 24 lags and a p-value of 0.1266. Alltough higher, this result is still acceptable.\\